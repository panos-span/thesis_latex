\chapter*{\centering Abstract}
Evaluating scholarly impact accurately is crucial for academic advancement, but traditional metrics 
like the H-Index have limitations, particularly their susceptibility to inflation through 
lower-quality publications.  This thesis proposes three novel H-Index variants—H5-Adjusted H-Index,
JIF-Adjusted H-Index, and EigenFactor-Adjusted H-Index—that incorporate journal quality metrics
and subject-specificity to provide a more accurate measure of scholarly impact.

Data was collected from the Crossref-2024 dataset, and subject classifications were assigned 
using Scopus ASJC codes. The adjusted H-Indexes were calculated by considering only publications 
in the top 20\% of journals within specific subjects, based on their H5, JIF, and EigenFactor 
rankings. The implementation involved SQL and ROLAP analysis, with additional calculations performed 
in Python, and unit testing that ensured the accuracy and reliability of these calculations.

The study found strong correlations between the traditional H-Index and the adjusted H-Indexes, 
indicating that researchers who rank highly by traditional metrics also rank highly when journal 
quality is considered. Analysis revealed that authors publishing in lower-tier journals tend to 
cite less influential research compared to those publishing in top-tier journals. Additionally, 
citation network analysis showed that top authors in general are part of more interconnected and influential 
research networks.

These findings suggest that incorporating journal quality into H-Index calculations mitigates 
issues such as H-Index inflation and provides a more nuanced assessment of scholarly impact. 
This approach benefits researchers, practitioners, and policymakers by offering more equitable 
and accurate academic evaluations. Finally, the study highlights the need for further research into 
advanced H-Index variants and subject-specific citation patterns to enhance bibliometric evaluations.
