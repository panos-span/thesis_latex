\chapter{Methodology}
\label{ch:methods}

\section{Overview of the Proposed Method}

The proposed method aims to address the problem of h-index inflation by
developing two novel metrics that adjusts for the subject area and the quality
of the journal in which the publication is published. The first metric, the
\textit{Adjusted h-index} ($h_{\text{adj}}$), adjusts the h-index by the
average h-index of the subject area in which the publication is published and
only considers publications from the top 20\% of journals in the subject area.
The second metric, the \textit{JIF-adjusted h-index} ($h_{\text{JIF}}$),
adjusts the h-index by the average Journal Impact Factor (JIF) of the journal
in which the publication is published and, similarly, only considers
publications from the top 20\% of journals in the subject area. The proposed
method is implemented in SQL and the data is stored in a Sqlite database.

\section{Data Collection and Preprocessing}

For the purpose of this study, the Alexandria3k \cite{Spi23g} tool was used to
collect publication data from the Crossref Dataset \cite{Crossref2020} and
prepare the data for processing. Alexandria3k supplies a library and a
command-line tool providing efficient relational query access to diverse
publication open data sets. Specifically, the dataset that was used is a subset
of the Crossref-2024 and Crossref-2023 datasets, which contain publication
metadata from about 156 million publications from all major international
publishers with full citation data for around 60 million of them. In addition
to that, for the subject of the publications, the Scopus ASJC (All Science
Journal Classification) subject codes were used.

To be more specific, the ASJC codes were used to categorize the publications
into more general subject areas than the original subject areas provided by the
Crossref-2023 dataset. The ASJC codes are a standard classification system used
to categorize publications into subject areas and are widely used in the
scientific community. The ASJC codes are hierarchical and are organized into
four levels, with the first level being the broadest and the fourth level being
the most specific. For the purpose of this study, the first level of the ASJC
codes was used to categorize the publications into subject areas. Crucial to
note is that the ASJC codes are not available for all publications in the
Crossref-2024 dataset \cite{crossrefSubjectCodes2024}, so only journals with
subjects from the Crossref-2023 dataset were considered for this study, since
for those journals the ASJC codes were able to be retrieved.


\section{Implementation of the Proposed Method}

The proposed method was implemented in SQL using the Sqlite database management
system. The implementation consists of two main parts: the calculation of the
Adjusted h-index and the calculation of the JIF-adjusted h-index. The
implementation of the Adjusted h-index is done in these steps: (1)
calculating the average h-index of the subject area for each publication, (2)



---
Methodology involves the design of research developed to test hypotheses or
answer questions developed from the background section. Include details about
data collection methods, analysis techniques, and tools, and technologies used.
In systems-building research in this chapter you might also include your
system's design and architecture.

Be careful regarding terminology: You describe \emph{methods} (not
methodology). This description is termed \emph{methodology}.

Base your methodological presentation using the terms introduced in reference
\cite{SF18}.

