\chapter{Conclusion}
\label{ch:conclusion}

Summarize the main contributions of the research and how they address the
research questions or hypotheses. Reflect on the implications of the findings,
the limitations of the study, and suggests areas for future research.

\section{Summary of Contributions}

\section{Implications for Researchers and Practitioners}

\section{Limitations of the Study}

This study, while providing valuable insights into the effectiveness of
adjusted H-Indexes, has several limitations that must be acknowledged.

One significant limitation is the reliance on Crossref-2024 data, which did not
include subject classifications for the journals. To address this, we used the
subject classifications from the previous year’s data for the journals. This
approach assumes that the subject classifications of journals do not change
significantly from year to year, which may not always be the case.
Additionally, we supplemented missing subject data using the Scopus API, which
introduces another layer of dependency and potential inconsistencies. The
reliance on historical data and external APIs for subject classification could
lead to inaccuracies in the analysis, as some journals might have updated their
scope or focus areas, and new journals may not have been appropriately
classified.

Moreover, our study filtered the publications to include only those in the top
20\% of journals by subject rank. While this method ensures a focus on
high-quality publications, it may exclude significant contributions from
journals that do not rank in the top 20\% but are still reputable and impactful
within their respective fields. This filtering criterion might bias the results
towards researchers who publish predominantly in highly ranked journals,
potentially overlooking the contributions of those whose work is more niche or
interdisciplinary and published in lower-ranked but still respected journals.

Another limitation is the scope of H-Index variants considered. While we
developed and evaluated adjusted H-Indexes incorporating journal quality
metrics, more advanced H-Index variants could further address issues such as
self-citation and hyperauthorship, which are known to cause H-Index inflation.
Variants like the fractional H-Index, which distributes citation credit more
equitably among co-authors, or metrics that specifically account for
self-citations, could provide a more refined measure of scholarly impact. Our
study did not implement these more sophisticated variants, which might offer
even better adjustments for the limitations of the traditional H-Index.

In summary, while our study makes significant strides in refining the H-Index
to account for journal quality and subject specificity, it is constrained by
data limitations, the selection of H-Index variants, and the methods used for
evaluation. Addressing these limitations in future research could further
enhance the accuracy and fairness of bibliometric evaluations, providing a more
robust tool for assessing scholarly impact.