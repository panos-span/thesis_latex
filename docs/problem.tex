\chapter{Problem Definition and Main Idea}
\label{ch:problem}

\section{Definition of H-Index Inflation}
The phenomenon of inflation of the h-index is a significant concern in
contemporary scientific metrics, particularly driven by the rise of
hyperauthorship. It refers to the artificial boosting of a scholar's h-index
through various practices such as self-citation, extensive co-authorship, and
the use of certain databases, which arises from the consistent growth of
scientific publications and citations over time. The primary reason behind this
is that more recent publications have access to a larger pool of potential
citations, thereby increasing their citation counts more rapidly
\cite{norris2010h, koltun2021h, bi2023four}. Database choice also plays a
significant role in h-index inflation due to varying coverage of publications
and citations. Jacso highlighted the variability in h-index scores depending on
the database used (Google Scholar, Scopus, Web of Science), noting that
databases like Google Scholar tend to have higher citation counts due to their
broader and less selective indexing practices and Meho and Yang found that
combining citation counts from multiple databases (Scopus and Web of Science)
resulted in a higher overall citation count, which could inflate h-index scores
\cite{norris2010h}.

Moreover, the lack of subject specificity in the calculation of the H-Index
could significantly contribute to its inflation. Without taking into account
the prestige and quality of the journals in which the publications are
published, the H-Index can be artificially boosted by publications in
lower-quality journals, that may not have rigorous peer-review processes.
Specifically, researchers might target journals with lower standards, which can
increase the number of published papers and citations without corresponding
increases in the quality of research. This inflation can distort the assessment
of academic influence and undermine the credibility of bibliometric
evaluations.

Important to note is that the h-index inflation can also drive unethical
practices like gift authorship, where individuals add each other’s names to
publications with minimal or no contribution to artificially boost their
h-index \cite{bi2023four}. Finally, the failure to convert nominal citation
values into real citation values, which are adjusted for the growth of
scientific output, results in significant mismeasurement of scientific impact
which particularly affects the evaluation of the career impact of researchers
who started their careers at different times since researchers from earlier
generations are likely to have their scientific impact underestimated if their
citation counts are not adjusted for inflation \cite{petersen2019methods}.

\section{Impact of Journal Quality on H-Index Calculation}
Considering the quality of journals in which an author's works are published
can significantly enhance the accuracy and reliability of assessing their
scientific impact. Journal quality, often gauged by metrics such as the Journal
Impact Factor (JIF), which measures the importance of a journal by calculating
the number of times selected articles are cited within a particular year and
reputation within the academic community, plays a crucial role in how citations
are distributed and perceived \cite{garfield1999journal, garfield2006history}.
A higher impact factor for a journal indicates that more of the journal's
publications have been highly cited, which suggests a higher quality of
research being published in the journal. For example, a JIF of 1.0 means that,
on average, articles published one or two years ago have been cited one time,
while a JIF of 2.5 means that articles have been cited 2.5 times on average.
This can be explained by the fact that high-impact journals generate more
citations due to their wide readership, rigorous peer-review process, and the
prestigious nature of the publications they attract \cite{garfield2006history}.

Similarly, journals with higher h-index values tend to have a higher proportion
of highly cited papers, which indicates that the journal is publishing
high-quality research that is widely recognized and respected by the scientific
community. For example, a journal with a high h-index may have published a
significant number of papers that have been cited many times, indicating that
the research is highly influential and widely accepted. On the other hand, a
journal with a low h-index may have published fewer highly cited papers,
suggesting that the research is less influential or less well-received by the
scientific community.

More recently, the Eigenfactor metric has emerged as an alternative measure of
journal quality, which evaluates the influence of a journal based on the number
of incoming citations, with citations from highly influential journals weighted
more heavily than those from less influential journals. It was designed to
adress the limitations of traditional citation metrics like the Impact Factor
(JiF), which can be skewed by a few highly cited papers, by leveraging the
structure of citation networks.

In detail, the Eigenfactor is calculated by analyzing the network of academic
citations among journals. It employs an algorithm similar to Google's PageRank
to measure the importance of journals based on their citations and the process
involves creating a matrix that records how often each journal cites every
other journal. Using this matrix, the algorithm simulates a random walk through
the citation network, where a hypothetical researcher selects articles and
follows their citations to other journals repeatedly. This random walk model
helps determine the frequency with which a researcher would visit each journal,
reflecting its overall influence. Finally, the Eigenfactor Score is derived
from these visitation frequencies, giving more weight to citations from
prestigious journals, and adjusting for differences in citation practices
across various fields \cite{Bergstrom11433, Alan2009}.

Having the above in mind, when evaluating an author's h-index, it could be
beneficial to focus on their publications in reputable journals with high
impact factors, eigenfactors or h-index values, which helps to exclude
publications of questionable quality and provides a more accurate and nuanced
estimation of the author's true impact. Thus, by prioritizing high-quality
journals, the h-index calculation better reflects the significance and
recognition of the author's contributions to their field.

\section{Research Questions and Hypotheses}
This thesis, by introducing a new metric that incorporates journal quality into
the calculation of the H-Index, with different ways of ranking journals based
on their quality, aims to address the limitations of the traditional H-Index
and provide a more accurate and meaningful assessment of a researcher's
scholarly. As such, the following research questions are posed: First, what is
the correlation between traditional H-Index values and the proposed H-Index
metrics that account for journal quality? Second, how do citation practices
differ between top authors publishing in low-quality journals and authors
publishing in high-quality journals. Third, how do citation patterns differ
between hyperprolific researchers and regular researchers across different
subject areas? By addressing these research questions, this thesis seeks not
only to develop and validate improved H-Index calculations that incorporate
journal quality, but to also provide insights into the citation practices and
patterns of researchers across different subjects, thus contributing to a more
complete understanding of scholarly impact.
