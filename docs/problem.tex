\chapter{Problem Definition and Main Idea}
\label{ch:problem}

\section{Definition of H-Index Inflation}
The phenomenon of inflation of the h-index is a significant concern in
contemporary scientific metrics, particularly driven by the rise of
hyperauthorship. It refers to the artificial boosting of a scholar's h-index
through various practices such as self-citation, extensive co-authorship, and
the use of certain databases, which arises from the consistent growth of
scientific publications and citations over time. The primary reason behind this
is that more recent publications have access to a larger pool of potential
citations, thereby increasing their citation counts more rapidly
\cite{norris2010h, koltun2021h, bi2023four}. Database choice also plays a
significant role in h-index inflation due to varying coverage of publications
and citations. Jacso highlighted the variability in h-index scores depending on
the database used (Google Scholar, Scopus, Web of Science), noting that
databases like Google Scholar tend to have higher citation counts due to their
broader and less selective indexing practices and Meho and Yang found that
combining citation counts from multiple databases (Scopus and Web of Science)
resulted in a higher overall citation count, which could inflate h-index scores
\cite{norris2010h}.

Moreover, the lack of subject specificity in the calculation of the H-Index
could significantly contribute to its inflation. Without taking into account the
subject-specific impact of journals, the H-Index can be artificially boosted by
publications in lower-quality journals. Researchers might target journals with
lower standards, which can increase the number of published papers and
citations without corresponding increases in the quality of research. This
inflation can distort the assessment of academic influence and undermine the
credibility of bibliometric evaluations.

Important to note is that the h-index inflation can also drive unethical practices
like gift authorship, where individuals add each other’s names to publications
with minimal or no contribution to artificially boost their h-index
\cite{bi2023four}. Finally, the failure to convert nominal citation values into
real citation values, which are adjusted for the growth of scientific output,
results in significant mismeasurement of scientific impact which particularly
affects the evaluation of the career impact of researchers who started their
careers at different times since researchers from earlier generations are
likely to have their scientific impact underestimated if their citation counts
are not adjusted for inflation \cite{petersen2019methods}.

\section{Impact of Journal Quality on H-Index Calculation}
Considering the quality of journals in which an author's works are published
can significantly enhance the accuracy and reliability of assessing their
scientific impact. Journal quality, often gauged by metrics such as the Journal
Impact Factor (JIF), which measures the importance of a journal by calculating
the number of times selected articles are cited within a particular year and
reputation within the academic community, plays a crucial role in how citations
are distributed and perceived \cite{garfield1999journal, garfield2006history}.
A higher impact factor for a journal indicates that more of the journal's
publications have been highly cited, which suggests a higher quality of
research being published in the journal. For example, a JIF of 1.0 means that,
on average, articles published one or two years ago have been cited one time,
while a JIF of 2.5 means that articles have been cited 2.5 times on average.
This can be explained by the fact that high-impact journals generate more
citations due to their wide readership, rigorous peer-review process, and the
prestigious nature of the publications they attract \cite{garfield2006history}.

Similarly, journals with higher h-index values tend to have a higher proportion
of highly cited papers, which indicates that the journal is publishing
high-quality research that is widely recognized and respected by the scientific
community. For example, a journal with a high h-index may have published a
significant number of papers that have been cited many times, indicating that
the research is highly influential and widely accepted. On the other hand, a
journal with a low h-index may have published fewer highly cited papers,
suggesting that the research is less influential or less well-received by the
scientific community.

Having the above in mind, when evaluating an author's h-index, it could be
beneficial to focus on their publications in reputable journals with high
impact factors or h-index values, which helps to exclude publications of
questionable quality and provides a more accurate and nuanced estimation of the
author's true impact. Thus, by prioritizing high-quality journals, the h-index
calculation better reflects the significance and recognition of the author's
contributions to their field.

\section{Research Questions and Hypotheses}
This thesis aims to answer several key research questions to evaluate the
proposed H-Index metrics. First, how does incorporating subject-specific
journal rankings based on H5-Index and impact factor affect the H-Index
calculation? Second, what is the correlation between traditional H-Index values
and the proposed H-Index metrics that account for journal quality? Third, how
do citation patterns differ between researchers with high traditional H-Indexes
and those with high proposed H-Indexes that incorporate journal quality? By
addressing these research questions, this thesis seeks to develop and validate
improved H-Index calculations that incorporate journal quality, providing a
more accurate and meaningful assessment of a researcher's scholarly impact.
