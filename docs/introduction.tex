\chapter{Introduction}
\label{ch:intro}

\section{Background and Motivation}
In the realm of academic research, evaluating the impact and contribution of
scholars is pivotal for numerous purposes, including tenure decisions, grant
allocations, and academic promotions. One of the most widely used metrics for
this evaluation is the H-Index, which aims to measure both the productivity and
citation impact of a scholar’s publications. Despite its widespread adoption,
the H-Index has been critiqued for several limitations, particularly its
susceptibility to inflation through the inclusion of lower-quality publications
\cite{costas2007h, tonta2020monetary}.

According to Hirsch, a scientist has an index \( h \) if \( h \) of his or her
\( N_p \) papers have at least \( h \) citations each, and the other \( (N_p -
h) \) papers have \(\leq h \) citations each.

\[
      \text{H-Index} = \max \left( h : \text{there are at least } h \text{ papers with } h \text{ or more citations each} \right)
\]

Formally, for a given researcher with \( N_p \) papers, the H-Index \( h \) is
defined as:

\[
      h = \max \left\{ h \, \Bigg| \, \sum_{i=1}^{N_p} \mathbf{1} (c_i \geq h) \geq h \right\}
\]

where:
\begin{itemize}
      \item \( N_p \) is the total number of papers,
      \item \( c_i \) is the number of citations for the \( i \)-th paper,
      \item \( \mathbf{1}(\cdot) \) is the indicator function, which is 1 if the condition inside is true and 0 otherwise.
\end{itemize} In simpler terms, the H-Index is defined as the maximum value of h
such that the given author has published h papers that have each been cited at least h times \cite{hirsch2005index}.
While straightforward, this metric does not account for the quality of the journals where
these papers are published. Consequently, an author could achieve a high H-Index
by publishing numerous papers in lower-impact journals, potentially inflating their perceived impact \cite{tonta2020monetary}.

This inflation problem becomes particularly problematic in interdisciplinary
fields where the impact factor and journal quality can vary significantly
across different subjects \cite{hirsch2005index,waltman2012inconsistency}.
Therefore, there is a pressing need to refine the H-Index by incorporating
metrics that reflect journal quality and subject specificity. Addressing these
limitations could enhance the accuracy and fairness of academic evaluations,
ensuring that the metric truly reflects the scholarly impact of researchers.

\section{Problem Statement}
The traditional H-Index has significant limitations affecting its reliability
and accuracy as a measure of scholarly impact. Primarily, as mentioned, it
fails to differentiate between the quality of journals where papers are
published, allowing authors to achieve high H-Indexes through publications in
lower-quality journals, thereby inflating their perceived impact
\cite{costas2007h,tonta2020monetary}. This issue is exacerbated in
interdisciplinary research, where measured journal quality can vary widely
across different subjects, due to variations in citation practices
\cite{costas2007h,norris2010h}.

Specifically, H-Index inflation occurs when researchers achieve higher H-Index
values not necessarily through impactful research but through strategic
publication practices. These practices may include targeting journals with
lower standards, which can increase the number of published papers and
citations without corresponding increases in the quality of research. This
inflation can distort the assessment of academic influence and undermine the
credibility of bibliometric evaluations \cite{tonta2020monetary}.

Another critical limitation of the traditional H-Index is its susceptibility to
self-citations and collaborative publications. Researchers may engage in
self-citation practices to boost their citation counts, which directly impacts
their H-Index
\cite{hirsch2005index,schreiber2008share,costas2007h,waltman2012inconsistency,norris2010h,bartneck2011detecting,zhivotovsky2008self}.
Similarly, in fields with high levels of collaboration, the H-Index may be
inflated as it does not account for the fractional contribution of each author
in multi-authored papers along with the citation practices of the authors
\cite{schreiber2008share,costas2007h,waltman2012inconsistency,norris2010h,zhivotovsky2008self}.
These factors can lead to significant disparities in H-Index values that do not
accurately reflect individual researchers' contributions and impact.

To address these issues, this thesis proposes developing and evaluating H-Index
calculations that incorporate subject-specific journal rankings based on
H5-Index and impact factor. By focusing on the top 20\% of journals within each
subject area, the goal is to create a more accurate and meaningful metric that
reflects both the productivity and the quality of an author's research
contributions.

\section{Objectives of the Thesis}
The primary objective of this thesis is to develop and validate H-Index
calculations that incorporate journal quality through subject-specific
rankings. The specific objectives are as follows:
\begin{enumerate}
      \item \textbf{Develop H-Index Metrics:} Create two H-Index metrics that incorporate journal quality
            by using subject-specific rankings based on H5-Index and journal impact factor (JIF).
            This involves selecting the top 20\% of journals within each subject area and calculating the H-Index based on publications
            in these high-quality journals.
            %\item \textbf{Evaluate the Impact of Journal Quality:} Analyze how incorporating journal quality affects the H-Index calculation.
            %      This includes comparing the traditional H-Index with the proposed H-Indexes to determine the extent to which the latter
            %      provide a more accurate representation of a scholar’s impact.
      \item \textbf{Assess Correlation Between Traditional H-Index and Proposed H-Indexes:} Assess the correlation between the traditional H-Index
            and the proposed H-Indexes. This involves statistical analysis by rank order correlation
            to evaluate the consistency and divergence in author rankings when different criteria for journal quality are applied.
      \item \textbf{Analyze Citation Practices and conduct Network Analysis of Citation Patterns:} Analyze citation practices of authors publishing in high- and low-ranked journals
            and examine the citation graphs of highly ranked authors based on their publications in lower-ranked versus
            higher-ranked journals within their specific subjects. The analysis will focus on network characteristics such as
            clustering coefficients and citation patterns. % TODO: CHANGE AFTER BEING DONE
            %\item \textbf{Compare Citation Behaviors Across Journal Quality:} Compare citation behaviors and network characteristics
            %      of authors publishing in the bottom 50\% of journals with those publishing in higher-ranked journals.
            %      This comparison aims to understand the differences in citation patterns and the implications for using
            %      citation network analysis in evaluating research impact.
      \item \textbf{Provide Practical Recommendations:} Offer practical recommendations for academic institutions,
            funding agencies, and researchers. These recommendations will focus on the use of H-Indexes and the importance of
            considering journal quality in research evaluations, aiming to promote more equitable and accurate assessments of scholarly impact.
\end{enumerate}

\section{Structure of the Thesis}
This thesis is organized into six main chapters, each addressing specific
aspects of the research and providing a comprehensive framework for
understanding and evaluating the proposed H-Index calculations.
\begin{itemize}
      \item \textbf{Chapter 1: Introduction} \\
            The Introduction chapter sets the stage for the thesis by providing the background and motivation for the study, outlining the problem statement, and defining the objectives of the research. This chapter also includes the structure of the thesis to guide the reader through the subsequent sections.

      \item \textbf{Chapter 2: Literature Review} \\
            The Literature Review chapter provides an extensive overview of the H-Index and its various popular variants, such as the H-Frac Index, Hp-Index, Hp-Frac Index, G-Index, O-Index, and J-Index. This chapter includes a comparative analysis of these metrics, highlighting their strengths, limitations, and applications in different disciplines.

      \item \textbf{Chapter 3: Problem Definition and Main Idea} \\
            The Problem Definition and Main Idea chapter clearly defines the problem of H-Index inflation, emphasizing the impact of journal quality on H-Index calculations. This chapter introduces the main idea of using subject-specific journal rankings to develop H-Indexes. The research questions and hypotheses are outlined, providing a focused framework for the study.

      \item \textbf{Chapter 4: Detailed Description of the Proposed Solution} \\
            This chapter presents a comprehensive description of the proposed solution.
            It begins with an overview of the methodology and proceeds to detailed subsections
            on data collection and preparation using Alexandria3k, ROLAP analysis with Simple-ROLAP,
            and SQL unit testing with RDBUnit. The chapter elaborates on the implementation of H-Indexes,
            including the use of the top 20\% of journals by H5-Index and impact factor within specific subjects.

      \item \textbf{Chapter 5: Results} \\
            The Results chapter presents the findings of the research.
            It includes a detailed analysis of the effectiveness of the adjusted H-Indexes,
            examining the correlation between traditional and proposed H-Indexes.
            This chapter also includes a citation practices analysis and a network
            analysis of citation patterns, highlighting the differences in citation networks between
            top and random authors by subject. % TODO: CHANGE AFTER BEING DONE!!

      \item \textbf{Chapter 6: Discussion and Implications} \\
            The Discussion and Implications chapter interprets the findings of the study,
            emphasizing the significance of the adjusted H-Indexes and their correlation
            with the traditional H-Index. It explores the implications for researchers and practitioners,
            suggesting that publishing in high-quality journals enhances the accuracy of scholarly impact assessments.
            The chapter also discusses the need for further research into more nuanced H-Index variants that address issues
            like self-citation and hyperauthorship. Finally, it showcases the methodological approach used in the study,
            advocating for the adoption of similar methodologies in future bibliometric evaluations.

      \item \textbf{Chapter 7: Conclusion} \\
            The Conclusion chapter summarizes the main contributions of the research,
            reflects on the implications of the findings, discusses the limitations of the study,
            and suggests areas for future research. It provides a comprehensive overview of the study's
            outcomes and their implications for the field of bibliometrics and research evaluation, along
            with the limitations of the study.
\end{itemize}

By following this structured approach, the thesis aims to provide a thorough
examination of the proposed H-Index implementations and their implications for
research evaluation, ultimately contributing to more accurate and equitable
academic assessments.

