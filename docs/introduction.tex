\chapter{Introduction}
\label{ch:intro}

\section{Background and Motivation}
In the realm of academic research, evaluating the impact and contribution of
scholars is pivotal for numerous purposes, including tenure decisions, grant
allocations, and academic promotions. One of the most widely used metrics for
this evaluation is the H-Index, which aims to measure both the productivity and
citation impact of a scholar’s publications. Despite its widespread adoption,
the H-Index has been critiqued for several limitations, particularly its
susceptibility to inflation through the inclusion of lower-quality publications
\cite{costas2007h, tonta2020monetary}.

According to Hirsch, a scientist has an index \( h \) if \( h \) of his or her
\( N_p \) papers have at least \( h \) citations each, and the other \( (N_p -
h) \) papers have \(\leq h \) citations each.

\[
      \text{H-Index} = \max \left( h : \text{there are at least } h \text{ papers with } h \text{ or more citations each} \right)
\]

Formally, for a given researcher with \( N_p \) papers, the H-Index \( h \) is
defined as:

\[
      h = \max \left\{ h \, \Bigg| \, \sum_{i=1}^{N_p} \mathbf{1} (c_i \geq h) \geq h \right\}
\]

where:
\begin{itemize}
      \item \( N_p \) is the total number of papers,
      \item \( c_i \) is the number of citations for the \( i \)-th paper,
      \item \( \mathbf{1}(\cdot) \) is the indicator function, which is 1 if the condition inside is true and 0 otherwise.
\end{itemize} In simpler terms, the H-Index is defined as the maximum value of $h$
such that the given author has published $h$ papers that have each been cited at least $h$ times \cite{hirsch2005index}.
While straightforward, this metric does not account for the quality of the journals where
these papers are published. Consequently, an author could achieve a high H-Index
by publishing numerous papers in lower-impact journals, potentially inflating their perceived achievements \cite{tonta2020monetary}.

This inflation phenomenon becomes particularly problematic in interdisciplinary
fields where the impact and quality of journals can vary significantly across
different subjects \cite{hirsch2005index,waltman2012inconsistency}. Therefore,
there is a pressing need to refine the H-Index by incorporating metrics that
reflect journal quality and subject specificity. Addressing these limitations
could enhance the accuracy and fairness of academic evaluations, ensuring that
the metric truly reflects the influence of researchers.

\section{Problem Statement}
The phenomenon of inflation of the h-index is a concern of high importance in
contemporary scientific metrics. It refers to the artificial boosting of a
scholar's h-index through various practices such as self-citation, extensive
co-authorship, and the use of certain databases, which arises from the
consistent growth of scientific publications and citations over time. The
primary reason behind this is that more recent publications have access to a
larger pool of potential citations, thereby increasing their citation counts
more rapidly \cite{norris2010h, koltun2021h, bi2023four}. Database choice also
plays a role in h-index inflation due to varying coverage of publications and
citations. Jacso highlighted the variability in h-index scores depending on the
database used (Google Scholar, Scopus, Web of Science), noting that databases
like Google Scholar tend to have higher citation counts due to their broader
and less selective indexing practices, and Meho and Yang found that combining
citation counts from multiple databases (Scopus and Web of Science) resulted in
a higher overall citation count, which could inflate h-index scores
\cite{norris2010h}.

Crucial to acknowledge is that, a highly important aspect of the H-Index
inflation is the lack of consideration of the quality of the journals in which
the publications are published. Without taking into account the prestige and
quality of the journals, the H-Index can be artificially boosted by
publications in lower-quality journals, that may not have rigorous peer-review
processes. Specifically, researchers might target those with lower standards,
which can increase the number of published papers and citations without
corresponding increases in the quality of research. This inflation can distort
the assessment of academic influence and undermine the credibility of
bibliometric evaluations \cite{tonta2020monetary}.

Additionally, the h-index inflation can also drive unethical practices like
gift authorship, where individuals add each other’s names to publications with
minimal or no contribution to artificially boost their h-index
\cite{bi2023four}. Finally, the failure to convert nominal citation values into
real citation values, which are adjusted for the growth of scientific output,
results in mismeasurement of scientific contribution, which particularly
affects the evaluation of the career impact of researchers who started their
careers at different times. This is because, researchers from earlier
generations are likely to have their achievements underestimated, if their
citation counts are not adjusted for inflation \cite{petersen2019methods}.

\section{Research Questions and Hypotheses}
This thesis, by introducing a new metric that incorporates journal quality into
the calculation of the H-Index, with different ways of ranking journals based
on their quality, aims to address the limitations of the H-Index and provide a
more accurate and meaningful assessment of a researcher's impact. As such, the
following research questions are posed: First, what is the correlation between
traditional H-Index values and the proposed H-Index metrics that account for
journal quality and what does that mean for the assessment of authors? Second,
how do citation practices differ between top authors publishing in low-quality
journals and authors publishing in high-quality journals? Third, how do
citation patterns differ between hyperprolific and regular researchers across
different subject areas? By addressing these research questions, this thesis
seeks, not only, to develop and validate improved H-Index calculations that
incorporate journal quality, but to also provide insights into the citation
practices and patterns of researchers across different subjects.