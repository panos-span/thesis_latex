\chapter{Project/Activities}
\label{ch:project}

\section{Training / Data Migration Project}

During the short duration of my training at the department, I was assigned to a
Data Migration Project. The project aimed to migrate data from the existing
Excel spreadsheets to a new cloud data lake architecture in Azure using Azure
Data Factory and SQL Data Warehouse. The data migration process involved
extracting data from the source system, transforming it to fit the new data
model, and loading it into the data lake. The project was part of a larger
initiative to modernize the data infrastructure and improve data management
capabilities. During my training, I became familiar with the data migration
process, learned how to use Azure Data Factory and SQL Data Warehouse, and
gained some experience working with cloud-based data platforms. However, due to
the short duration of my training and my wishes to focus on a project that
would allow me to gain more hands-on experience with data analytics and machine
learning, I requested to be reassigned to a different project, thus I was
assigned to a Cable Coil Prediction AI Project.

\section{Cable Coil Prediction AI Project}

After discussing my interests and goals with my supervisor at the previous
project, I was reassigned a Cable Coil Prediction AI Project, which before me
the team consisted of 2 people, one being the manager of the project, and the
other being a data scientist. This project aimed to develop an AI-powered
solution for predicting the quality of produced cable coils. The project
involved collecting IoT time series data from cable production processes,
preprocessing the data, extracting relevant features, and developing predictive
models to analyze the data and predict the quality of the cable coils. The goal
of the project was to leverage machine learning techniques to improve the
accuracy of quality predictions and optimize the production process. In order
to achieve this goal, which was to not only predict the quality of the cable
coils but also to identify the key indicators of cable coil quality, I was
assigned the following tasks, with my colleague:
\begin{itemize}
    \item \textbf{Data Collection:} Collecting IoT time series data from cable production processes and storing it in a suitable format for analysis by communicating with the production team and IT department.
    \item \textbf{Exploratory Data Analysis:} Analyzing the collected data to identify patterns, trends, and anomalies that could provide insights into the quality of the cable coils, including the creation of visualizations and summary statistics, such as histograms, box plots, and correlation matrices.
    \item \textbf{Data Preprocessing:} Cleaning the data, handling missing values, and normalizing the data to prepare it for analysis.
    \item \textbf{Feature Engineering:} Extracting relevant features from the raw data to improve the accuracy of the predictive models using domain knowledge and specialized time series feature extraction techniques by utilizing python libraries such as \emph{Tsfresh} and \emph{Sktime}
    \item \textbf{Model Development:} Developing predictive models using machine learning algorithms to analyze the time series data and predict the quality of the cable coils.
    \item \textbf{Model Evaluation:} Experimenting with various machine learning algorithms, such as Random Forest, Gradient Boosting, and Neural Networks, to determine the most effective approach.
    \item \textbf{Hyperparameter Tuning:} Tuning the hyperparameters of the machine learning models to optimize their performance and improve the accuracy of the predictions, by using specialized tools in Python such as \emph{Optuna} and \emph{GridSearchCV}.
    \item \textbf{Documentation and Reporting:} Documenting the data analysis process, model development, and evaluation results, and preparing reports and presentations to communicate the findings to the project team and stakeholders.
\end{itemize}
