\chapter{Description}
\label{ch:description}

\section{Department}

% Info about the department
The Data \& AI Department of EY is responsible for providing data analytics and
artificial intelligence services to clients. The department is staffed with
professionals with expertise in data science, and artificial intelligence. The
department's services include data analysis, data visualization, machine
learning model development, and other artificial intelligence solutions, such
as Chatbots and AI Assistants. The department works with clients from various
industries, including primarily financial services, telecommunications, retail
and public sector organizations. Additionally, the department provides it
services to other sectors such as energy, utilities, and healthcare. The
department's goal is to help clients leverage their data to make informed
business decisions, improve operational efficiency, and drive innovation, by
not only providing data analytics and artificial intelligence solutions but
also by offering strategic advice on how to best utilize data and AI
technologies and offering valuable insights on how to implement data-driven
strategies, in order to help clients understand their data and make better
decisions.

The Data \& AI Department at EY is structured to support its diverse range of
services and client needs. The department consists of approximately 200
professionals and the Department is led by the Leading Partner of the Data \&
AI Department, Elias Vyzas, who oversees all activities and ensures alignment.
Following the Leading Partner, the department is leaded by Directors, who are
responsible for the overall strategy and operations of the department along
with the Senior Managers. Additionally, the department is divided into several
teams, each specializing in a specific project or service area. The teams
include a Manager and the team members, who are responsible for the day-to-day
operations and project delivery. Important to note is that a manager and a team
member can be part of multiple teams, depending on the projects, the needs of a
project and the services that are provided to the clients.

\section{Position}

% Info about the position
The position I was assigned to during my internship at EY was that of a Data
Scientist. As a Data Scientist, my role was to support the Data \& AI
Department in delivering data analytics and artificial intelligence solutions
to clients. My responsibilities included working on data analysis, developing
machine learning models, and creating data visualizations to help clients
understand their data and make informed decisions. Specifically, I was involved
in an AI project for a client (who will remain unknown for confidentiality
reasons), which aimed to develop an AI-powered solution for predicting the
quality of produced cable coils.

My responsibilities included collecting and preprocessing IoT time series data
from cable production processes. This involved cleaning the data, handling
missing values, and normalizing the data to prepare it for analysis. I worked
on extracting relevant features from the raw data to improve the accuracy of
the predictive models, identifying key indicators of cable coil quality, and
transforming the data into a suitable format for machine learning algorithms.
Moreover, I developed predictive models using anisotropic learning techniques
to analyze the time series data and predict the quality of the produced cable
coils. This involved experimenting with various machine learning algorithms,
such as Random Forest, Gradient Boosting, and Neural Networks, to determine the
most effective approach.

I evaluated the performance of the predictive models using metrics such as
accuracy, precision, recall, AUC (Area Under Curve), and F1-score. Based on the
evaluation results, I optimized the models by tuning hyperparameters and
improving feature selection. I created interactive data visualizations to
present the findings and insights from the predictive models to stakeholders.
These visualizations helped clients understand the factors affecting cable coil
quality and make informed decisions to improve production processes. I
documented the methodologies, procedures, and results of the project in
detailed reports, including writing technical documentation and creating
presentations for both technical and non-technical audiences. I also
communicated regularly with clients to gather requirements, provide updates,
and deliver final solutions.

\section{Required Skills}

% Info about the required skills
The position of Data Scientist at Ernst \& Young Greece required a combination
of technical and soft skills to successfully deliver data analytics and
artificial intelligence solutions to clients. The technical skills required for
the position included proficiency in programming languages such as Python, SQL
and scripting languages such as Bash, as well as experience with data analysis
libraries such as Pandas, NumPy, and Scikit-learn. Additionally, knowledge of
machine learning algorithms, data preprocessing techniques, and model
evaluation methods was essential for developing predictive models and analyzing
data effectively as well as experience with time series analysis and feature
extraction techniques. Furthermore, experience with deep learning frameworks
such as TensorFlow and Keras was required to develop advanced machine learning
models for complex data analysis tasks.

The position also required experience with data visualization tools such as
Matplotlib, Seaborn, and Plotly to create interactive visualizations and
present insights to stakeholders. Moreover, familiarity with cloud platforms
such as Azure experience with version control systems such as Git was also
required to manage code and collaborate with team members effectively.

In addition to technical skills, the position required strong analytical and
problem-solving skills to identify patterns, trends, and anomalies in data and
develop innovative solutions to complex business problems. Effective
communication skills were also essential for presenting findings to clients,
collaborating with team members, and documenting project methodologies and
results. Lastly, the position required the ability to work independently and as
part of a team, manage multiple tasks simultaneously, and adapt to changing
project requirements and priorities.

\section{Expected Results}

The expected results of my internship at EY included successfully completing
the AI project and delivering a high-quality AI-powered solution for predicting
the quality of produced cable coils to the client. Apart from the technical
deliverables, I also aimed to satisfy the expectations of my supervisors and
colleagues by demonstrating professionalism, teamwork, and effective
communication throughout the internship.