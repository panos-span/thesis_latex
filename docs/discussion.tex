\chapter{Discussion and Implications}
\label{ch:discussion}

\section{Discussion of Results}

The results of the study show that the adjusted H-Indexes are significantly
correlated with the traditional H-Index, with correlation coefficients in the
range of 0.6 to 0.7. This indicates a strong positive relationship between the
traditional H-Index and the adjusted H-Indexes, suggesting that researchers who
rank highly according to the traditional H-Index also tend to rank highly
according to the adjusted H-Indexes.

However, compared to other H-Index variants, as reviewed in the literature, the
correlation is not perfect. This imperfection highlights the distinctiveness of
the adjusted H-Indexes. While the strong correlation signifies that the
adjusted H-Indexes retain the fundamental characteristics of the traditional
H-Index, the differences suggest that the adjustments made for journal quality
and subject specificity introduce meaningful variations. These variations
likely reflect the impact of publishing in higher-quality journals and the
influence of subject-specific citation practices, which are not captured by the
traditional H-Index.

The results from the citation analysis of the authors reveal that authors with
high H-Indexes from the top 20\% of journals by rank tend to have a higher
clustering coefficient, indicating that their works are more likely to be part
of tightly-knit citation clusters compared to other works with the same number
of citations. This suggests that highly cited authors' research is
interconnected within their respective fields, contributing to the formation of
citation clusters around their works. Additionally, the same applies with even
more significance to authors with high H-Indexes from the bottom 50\% of
journals by rank, indicating that the journal ranking significantly influences
the formation of citation clusters.

The results of the study provide valuable insights into the effectiveness of
adjusted H-Indexes and the impact of journal quality on the calculation of
scholarly impact metrics. By incorporating subject-specific journal rankings
based on H5-Index and impact factor, the adjusted H-Indexes offer a more
nuanced and accurate assessment of a researcher's impact, accounting for the
quality of the journals in which their works are published. The correlation
between traditional H-Index values and the proposed H-Index metrics that
account for journal quality highlights the importance of considering journal
quality in evaluating scholarly impact. The differences in citation patterns
between researchers with high traditional H-Indexes and those with high
proposed H-Indexes underscore the influence of journal quality and subject
specificity on scholarly impact metrics.

\section{Implications for Researchers and Practitioners}

The findings of this study have several implications for researchers and
practitioners in the field of bibliometrics and research evaluation. First,
researchers should be aware of the limitations of traditional H-Index metrics
and consider the impact of journal quality on their scholarly impact. By
publishing in reputable journals with high impact factors and h-index values,
researchers can enhance the visibility and recognition of their work, leading
to a more accurate assessment of their scholarly impact. Researchers could also
modify the threshold of the journal quality metrics, such as the top 10\% or
top 30\% of journals, to further refine the adjusted H-Indexes and tailor them
to their specific research field, based on how strict they want to be in
considering the journal quality.

Practitioners involved in research evaluation and funding decisions should also
take into account the quality of the journals in which researchers publish when
assessing their impact. By incorporating journal quality metrics into
evaluation criteria, practitioners can ensure a more comprehensive and
meaningful evaluation of researchers' contributions to their field.

Furthermore, the study highlights the need for further research into more nuanced
H-Index variants that address issues such as self-citation and hyperauthorship.
By developing more sophisticated, but not complex to calculate, H-Index metrics
that provide a more refined measure of scholarly impact, researchers and
practitioners can improve the accuracy and fairness of bibliometric
evaluations.

Lastly, the study showcases a methodological approach to evaluating scholarly
impact metrics that incorporate journal quality and subject specificity into
the calculation of H-Indexes. Specifically, by using the Alexandria3k tool to
extract and analyze publication data from Crossref-2024, along with the
effective use of ROLAP analysis and testing, the study demonstrates a robust
and reliable method for calculating efficiently and accurately adjusted
H-Indexes. By adopting similar methodologies, researchers and practitioners can
enhance the reliability and validity of bibliometric evaluations along with the
assessment of scholarly impact and much more.

%This thesis aims to answer several key research questions to evaluate the
%proposed H-Index metrics. First, how does incorporating subject-specific
%journal rankings based on H5-Index and impact factor affect the H-Index
%calculation? Second, what is the correlation between traditional H-Index values
%and the proposed H-Index metrics that account for journal quality? Third, how
%do citation patterns differ between researchers with high traditional H-Indexes
%and those with high proposed H-Indexes that incorporate journal quality? By
%addressing these research questions, this thesis seeks to develop and validate
%improved H-Index calculations that incorporate journal quality, providing a
%more accurate and meaningful assessment of a researcher’s scholarly impact.

The study answers all the research questions and hypotheses, and the results
show that the firstly, the incorporation of subject-specific journal rankings
based on H5-Index and impact factor significantly affects the H-Index
calculation, as it takes into account the quality of the journals in which the
researcher’s works are published, which in turn provides a more accurate and
nuanced estimation of the researcher’s true impact. Secondly, the correlation
between traditional H-Index values and the proposed H-Index metrics that
account for journal quality is strong, indicating a positive relationship
between the traditional H-Index and the adjusted H-Indexes. Finally, the
citation patterns differ between researchers with high traditional H-Indexes
and those with high proposed H-Indexes that incorporate journal quality,
highlighting the influence of journal quality and subject specificity on
scholarly impact metrics.

Present the findings of the research, often in detail, and discuss these
results in the context of the literature review to explain their relevance and
impact. Describe the implications of your research for researchers and
practitioners. Include a critical evaluation of the research.
