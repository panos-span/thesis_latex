\chapter{Discussion and Implications}
\label{ch:discussion}

\section{Discussion of Results}

The results of the study show that the adjusted H-Indexes are significantly
correlated with the traditional H-Index, with correlation coefficients in the
range of 0.6 to 0.7. This indicates a strong positive relationship between the
traditional H-Index and the adjusted H-Indexes, suggesting that researchers who
rank highly according to the traditional H-Index also tend to rank highly
according to the adjusted H-Indexes.

However, compared to other H-Index variants, as reviewed in the literature, the
correlation is not perfect. This imperfection highlights the distinctiveness of
the adjusted H-Indexes. While the strong correlation signifies that the
adjusted H-Indexes retain the fundamental characteristics of the traditional
H-Index, the differences suggest that the adjustments made for journal quality
and subject specificity introduce meaningful variations. These variations
likely reflect the impact of publishing in higher-quality journals and the
influence of subject-specific citation practices, which are not captured by the
traditional H-Index. In addition to that, the adjusted H-Indexes reduce the
number of authors with high H-Indexes compared to the traditional H-Index,
leading to disinflation of H-Index values and potentially providing a more
accurate representation of the authors' scholarly impact, which is evident from
the reduction in the number of authors with H-Indexes above certain thresholds,
as shown in Table \ref{tab:thresholds}. % TODO: Check again

The analysis of citation practices reveals substantial differences between
authors publishing in lower-tier journals and those publishing in top-tier
ones. Authors from lower-tier journals have a lower average h-index of cited
journals compared to random authors from top-tier journals, suggesting that
they cite less influential or lower-quality research. This pattern is
consistent across almost all subject areas, emphasizing the importance of
journal quality in citation practices and the impact on the overall scholarly
contribution of authors.

Furthermore, the citation network clustering analysis highlights that top
authors are more likely to be part of tightly-knit citation clusters. This
interconnectedness is particularly pronounced in fields like Physical Sciences
and Multidisciplinary research, where collaboration is more prevalent. The
higher clustering coefficients for top authors suggest that these authors are
more engaged in collaborative research and are part of a more cohesive
community, which may contribute to their higher h-index values.

These findings emphasize the need for incorporating journal quality and the
need for considering subject-specific citation network structures into
bibliometric evaluations. By doing so, we can achieve a more accurate and
comprehensive assessment of a researcher's true impact, beyond what the
traditional H-Index alone can offer. This approach helps to mitigate issues
such as H-Index inflation and not accounting for the quality of the journals in
which researchers publish along with the subject-specific citation practices,
and thus provides a clearer picture of scholarly contributions regardless of
the field of research.

\section{Implications for Researchers and Practitioners}

The findings of this study have several implications for researchers and
practitioners in the field of bibliometrics and research evaluation. First,
researchers should be aware of the limitations of traditional H-Index metrics
and consider the impact of journal quality on such metrics. By focusing on
publishing in reputable journals with high impact factor, eigenfactor and
h-index values, researchers can enhance the visibility and recognition of their
work, leading to a more accurate assessment of their scholarly impact.

Practitioners involved in research evaluation and funding decisions should also
take into account the quality of the journals in which researchers publish when
assessing their impact. Researchers could modify the threshold of the journal
quality metrics, such as the top 10\% or top 30\% of journals, to further
refine the adjusted H-Indexes and tailor them to their specific research field,
based on how strict they want to be in considering the journal quality. By
incorporating metrics that consider the journal quality of publications into
evaluation criteria, practitioners can ensure a more comprehensive and
meaningful evaluation of researchers' contributions to their field.

Furthermore, the study highlights the need for further research into more
advanced H-Index variants that address issues such as self-citation and
hyperauthorship. By developing more sophisticated, but not too complex to
calculate, H-Index metrics that provide a more refined measure of scholarly
impact, practitioners can improve the accuracy and fairness of bibliometric
evaluations.

Lastly, the study showcases a methodological approach for evaluating scholarly
impact metrics that incorporate journal quality and subject specificity into
the calculation of H-Indexes. Specifically, by using the Alexandria3k tool to
extract and analyze publication data from Crossref-2024, along with the
effective use of ROLAP analysis and unit testing, the study demonstrates a
robust and reliable method for calculating and analysing efficiently and
accurately adjusted H-Indexes. By adopting similar methodologies, researchers
can enhance the reliability and validity of bibliometric evaluations.

Overall, the study contributes to the ongoing discussion on the limitations of
traditional H-Index metrics and provides a more nuanced and accurate assessment
of scholarly impact by incorporating journal quality and subject specificity
into the evaluation criteria. By addressing these limitations and providing
insights into citation practices and patterns across different subject areas,
the study contributes to a more comprehensive understanding of scholarly impact
and research evaluation, ultimately benefiting researchers, practitioners, and
policymakers involved in academic assessments and funding decisions. %TODO: Check
