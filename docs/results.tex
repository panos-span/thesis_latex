\chapter{Results}
\label{ch:results}

\section{Statistical Analysis of the Adjusted H-indexes}

To evaluate the effectiveness of the adjusted h-indexes, we calculated the rank
order correlation between the h-index and the adjusted h-indexes from the
publications that were included in the top 20\% of the journals by rank. The
Pearson, Spearman, and Kendall tau rank order correlations were calculated for
the h-index and the adjusted h-indexes using the \emph{scipy.stats} package in
Python using Python 3.11. The results are shown in Table
\ref{tab:correlations}.

\begin{table}[H]
    \centering
    \begin{tabular}{|c|c|c|c|}
        \hline
        \textbf{Correlation}                            & \textbf{Pearson} & \textbf{Spearman} & \textbf{Kendall tau} \\ \hline
        \textbf{h-index vs. Adjusted h-index using H5}  & 0.686            & 0.69              & 0.637                \\ \hline
        \textbf{h-index vs. Adjusted h-index using JIF} & 0.65             & 0.66              & 0.6                  \\ \hline
    \end{tabular}
    \caption{Correlation between h-index and adjusted h-indexes}
    \label{tab:correlations}
\end{table}

%From these results, we can see that the adjusted h-indexes are significantly correlated with the
%h-index. However, compared to other H-index variants, as we have seen in the literature review, the
%correlation is not 

From these results, we can see that the adjusted H-Indexes are significantly
correlated with the traditional H-Index, with correlation coefficients in the
range of 0.7 to 0.8. This indicates a strong positive relationship between the
traditional H-Index and the adjusted H-Indexes, suggesting that researchers who
rank highly according to the traditional H-Index also tend to rank highly
according to the adjusted H-Indexes.

However, compared to other H-Index variants, as reviewed in the literature, the
correlation is not perfect. This imperfection highlights the distinctiveness of
the adjusted H-Indexes. While the strong correlation signifies that the
adjusted H-Indexes retain the fundamental characteristics of the traditional
H-Index, the differences suggest that the adjustments made for journal quality
and subject specificity introduce meaningful variations. These variations
likely reflect the impact of publishing in higher-quality journals and the
influence of subject-specific citation practices, which are not captured by the
traditional H-Index.

\section{Citation Analysis of the Authors}

A noteworthy observation was the number of authors with a high metric: the
top-ranked author has an h5-index of 76, x authors have an h5-index larger than
60, and x larger than 38. These achievements appear to be even more difficult
to explain than earlier observations of hyperprolific authors. It was found in
2022 that works by authors with a high h5-index (greater than 50) are more
likely to be part of tightly-knit citation clusters compared to other works
with the same number of citations, which can be explained by the fact that the
clustering coefficients of a random sample of 50 works from authors with an
h5-index larger than 50 appear to be significantly different from a random
sample of other works of the same size with the same number of citations for
each one (median 0.03). This highlights the interconnected nature of highly
cited authors' research within their respective fields \cite{Spi23g}.

We explored as well the phenomenal productivity and impact exhibited by the
authors at the top of the distribution (for the publications that were included
in the top 20\% of the journals by rank) for the adjusted h-indexes by
examining the clustering coefficient of the graph induced by incoming and
outgoing citations of distance 2 for a given work. For the h-index calculated
from the filtering of journals by rank using the H5-Index, the median
clustering coefficient of the first group is 0.05, \dots

For the h-index calculated from the filtering of journals by rank using the
JIF, the median clustering coefficient of the first group is 0.05, \dots

Additionally, in order to analyze further the effect of the journal ranking on
the formulation of tightly-knit citation clusters, we compared the clustering
coefficients of a random sample of 50 works from authors with an h5-index
larger than 10 to a random sample of other works of the same size with the same
number of citations for each one, but this time for the publications that were
included in the bottom 50\% of the journals by rank. For the h-index calculated
from the filtering of journals by rank using the H5-Index, the median
clustering coefficient of the first group is 0.05, \dots

For the h-index calculated from the filtering of journals by rank using the
JIF, the median clustering coefficient of the first group is 0.05, \dots

The results suggest that the clustering coefficients of the two groups are
significantly different, indicating that the journal ranking has a significant
impact on the formation of citation clusters.

The clustering coefficients of a random sample of 50 works from authors with an
h5-index larger than 50 (median coefficient 0.05) appear to be significantly
different from a random sample of other works of the same size with the same
number of citations for each one (median 0.03).

Present your research results in terms of statistical findings, empirical
evaluation or other methods you have used in your research.
