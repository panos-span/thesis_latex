\chapter{Results}
\label{ch:results}

\section{Statistical Analysis of the Adjusted H-indexes}

To evaluate the relationship of the adjusted h-indexes with the traditional
h-index, we calculated the rank order correlation between the h-index and the
adjusted h-indexes from the publications that were included in the top 20\% of
the journals by rank. The Pearson, Spearman, and Kendall tau rank order
correlations were calculated for the h-index and the adjusted h-indexes using
the \emph{scipy.stats} package in Python using Python 3.11.9. The results are
shown in Table \ref{tab:correlations}.

\begin{table}[H]
    \centering
    \renewcommand{\arraystretch}{1.5}
    \begin{tabular}{|>{\centering\arraybackslash}m{6cm}|>{\centering\arraybackslash}m{3cm}|>{\centering\arraybackslash}m{3cm}|>{\centering\arraybackslash}m{3cm}|}
        \hline
        \textbf{Correlation}                            & \textbf{Pearson} & \textbf{Spearman} & \textbf{Kendall tau} \\
        \hline
        \textbf{h-index vs. Adjusted h-index using H5}  & 0.686            & 0.693             & 0.637                \\
        \hline
        \textbf{h-index vs. Adjusted h-index using JIF} & 0.667            & 0.678             & 0.616                \\
        \hline
    \end{tabular}
    \caption{Correlation between h-index and adjusted h-indexes}
    \label{tab:correlations}
\end{table}

%From these results, we can see that the adjusted h-indexes are significantly correlated with the
%h-index. However, compared to other H-index variants, as we have seen in the literature review, the
%correlation is not 

From these results, we can see that the adjusted H-Indexes are significantly
correlated with the traditional H-Index, with correlation coefficients in the
range of 0.6 to 0.7. This indicates a strong positive relationship between the
traditional H-Index and the adjusted H-Indexes, suggesting that researchers who
rank highly according to the traditional H-Index also tend to rank highly
according to the adjusted H-Indexes. More specifically, the Pearson correlation
coefficients indicate a strong linear relationship, while the Spearman
correlation coefficients suggest a strong monotonic relationship. Lastly, the
Kendall tau correlation indicate a strong ordinal relationship between the
traditional H-Index and the adjusted H-Indexes.

Compared to other H-Index variants, as reviewed in the literature, the
correlation is not perfect. This imperfection highlights the distinctiveness of
the adjusted H-Indexes. While the strong correlation signifies that the
adjusted H-Indexes retain the fundamental characteristics of the traditional
H-Index, the differences suggest that the adjustments made for journal quality
and subject specificity introduce meaningful variations. These variations
likely reflect the impact of publishing in higher-quality journals and the
influence of subject-specific citation practices, which are not captured by the
traditional H-Index.

To further investigate the relationship between the traditional H-Index and the
adjusted H-Indexes, we calculated the number of authors with adjusted H-indexes
and traditional H-index above certain thresholds, along with their max values.
The results are shown in Table \ref{tab:thresholds}, where we can see that the
adjusted H-Indexes reduce the number of authors with high H-indexes compared to
the traditional H-Index, leading to disinflation of the H-index values and
potentially to a more accurate representation of the authors' scholarly impact.

\begin{table}[H]
    \centering
    \renewcommand{\arraystretch}{1.5}
    \begin{tabular}{|>{\centering\arraybackslash}m{4cm}|>{\centering\arraybackslash}m{3cm}|>{\centering\arraybackslash}m{3cm}|>{\centering\arraybackslash}m{3cm}|}
        \hline
        \textbf{Threshold} & \textbf{Adjusted H-Index using H5} & \textbf{Adjusted H-Index using JIF} & \textbf{Traditional H-Index} \\
        \hline
        $>40$              & 179                                & 179                                 & 213                          \\
        \hline
        $>50$              & 52                                 & 52                                  & 62                           \\
        \hline
        $>60$              & 19                                 & 19                                  & 20                           \\
        \hline
        \emph{Max}         & 87                                 & 87                                  & 98                           \\
        \hline
    \end{tabular}
    \caption{Number of authors with adjusted h-indexes and traditional h-index above certain thresholds}
    \label{tab:thresholds}
\end{table}

\section{Citation Practices of Authors}

In the process of analyzing citation practices, it is essential to compare
different groups of authors to understand their impact and engagement within
the academic community. This analysis focuses on two distinct groups: top
authors publishing in lower-tier journals and randomly selected authors
publishing in top-tier journals. By examining the h-index of the journals these
authors cite, we aim to uncover citation practises and provide insights into
the quality and influence of their scholarly work.

Specifically, we selected the top 50 authors per subject publishing in the
bottom 50\% of journals by rank and matched them with random authors publishing
in the top 20\% of journals, with matching h-indexes, where the H5 calculations
are stored in the tables \emph{orcid\_h5\_bottom} and
\emph{orcid\_h5\_filtered} accordingly. We then ensured one to one matching
based on the row number and calculated the average h-index of the cited
journals for each author in both groups.

being done with the above extraction we proceeded to obtain the Mann-Whitney U
statistic and p-value for the comparison of the average h-index of the cited
journals between the two groups per subject area. The results are shown in
tables \ref{tab:citation_practices_jif} and \ref{tab:citation_practices_h5}.

\begin{table}[H]
    \centering
    \renewcommand{\arraystretch}{1.5}
    \begin{tabular}{|>{\centering\arraybackslash}m{4cm}|>{\centering\arraybackslash}m{3cm}|>{\centering\arraybackslash}m{3cm}|>{\centering\arraybackslash}m{3cm}|>{\centering\arraybackslash}m{3cm}|}
        \hline
        \textbf{Subject Area}         & \textbf{Average for Top Authors} & \textbf{Average for Random Authors} & \textbf{Mann-Whitney U} & \textbf{P-Value} \\
        \hline
        Health Sciences               & 81.913                           & 88.812                              & 993                     & 0.077            \\
        \hline
        Life Sciences                 & 52.945                           & 85.091                              & 598                     & $<0.001$         \\
        \hline
        Multidisciplinary             & 30.064                           & 90.346                              & 191                     & $<0.001$         \\
        \hline
        Physical Sciences             & 44.642                           & 90.983                              & 154                     & $<0.001$         \\
        \hline
        Social Sciences \& Humanities & 26.703                           & 64.529                              & 294                     & $<0.001$         \\
        \hline
    \end{tabular}
    \caption{Citation Practices of Top Authors from Lower-Tier Journals and Random Authors from Top-Tier Journals by Subject Area, with ranking of the journals based on JIF}
    \label{tab:citation_practices_jif}
\end{table}

\begin{table}[H]
    \centering
    \renewcommand{\arraystretch}{1.5}
    \begin{tabular}{|>{\centering\arraybackslash}m{4cm}|>{\centering\arraybackslash}m{3cm}|>{\centering\arraybackslash}m{3cm}|>{\centering\arraybackslash}m{3cm}|>{\centering\arraybackslash}m{3cm}|}
        \hline
        \textbf{Subject Area}         & \textbf{Average for Top Authors} & \textbf{Average for Random Authors} & \textbf{Mann-Whitney U} & \textbf{P-Value} \\
        \hline
        Health Sciences               & 62.629                           & 87.214                              & 615                     & $<0.001$         \\
        \hline
        Life Sciences                 & 49.607                           & 81.686                              & 371                     & $<0.001$         \\
        \hline
        Multidisciplinary             & 41.781                           & 88.819                              & 224                     & $<0.001$         \\
        \hline
        Physical Sciences             & 42.835                           & 83.658                              & 204                     & $<0.001$         \\
        \hline
        Social Sciences \& Humanities & 32.427                           & 77.161                              & 353                     & $<0.001$         \\
        \hline
    \end{tabular}
    \caption{Citation Practices of Top Authors from Lower-Tier Journals and Random Authors from Top-Tier Journals by Subject Area, with ranking of the journals based on H5}
    \label{tab:citation_practices_h5}
\end{table}

The results of this analysis reveal significant differences in the citation
practices between the two groups of authors. The overall comparison shows that
the random authors, who publish in journals ranked in the top 20\%, have a
significantly higher average h-index of cited journals compared to the top
authors from the orcid\_h5\_bottom dataset. Specifically, the overall bottom
average h-index median is 47.91 and 47.50 (by H5 and JIF journal ranking
accordingly), while the overall random average h-index median is 82.9 and
84.85. The Mann-Whitney U test statistic is 8931.0 and 11842.0, with both an
extremely low p-value of $<0.001$, indicating a highly statistically
significant difference between the two groups.

The Mann-Whitney U test is a non-parametric test used to determine whether
there is a significant difference between the distributions of two independent
samples \cite{mann1947test}. In this context, the U value quantifies the
difference between the two groups. A higher U value generally indicates a
greater difference between the groups. The extremely low p-value associated
with the U value suggests that the observed difference is not due to random
chance, and there is a statistically significant difference between the two
groups.

Examining the results by subject by JIF journal ranking, a similar trend is
observed, with the exception of the Health Sciences, where the difference is
not statistically significant ($p-value = 0.077 > 0.05$), only being marginally
significant. In contrast, all other subject areas show highly significant
differences in the average h-index of cited journals between the two groups,
with p-values ranging from $7.08 \times 10^{-6}$ to $2.86 \times 10^{-13}$.
This suggests that top authors from lower-tier journals tend to cite journals
with lower h-indexes compared to random authors from top-tier journals,
indicating differences in the quality and impact of the publications they
reference.

Similarly, the results by subject by H5 journal ranking show a similar pattern,
but this time all subject areas show highly significant differences in the
average h-index of cited journals between the two groups, with p-values ranging
from $1.22 \times 10^{-5}$ to $5.70 \times 10^{-13}$. The difference between
the results of the two groups can be explained by the differences in
distribution of the H5 and JIF journal rankings, which influences the journals
that are considered high-quality and influential in each group, along with the
respective h-index values of the authors publishing in high and low tier
journals.

These findings suggest that authors publishing in lower-tier journals have
different citation practices compared to those publishing in top-tier journals.
The lower average h-index of cited journals among top authors from lower-tier
journals reflects a preference for citing less influential or lower-quality
research, potentially affecting the overall impact and credibility of their
work. In contrast, random authors from top-tier journals tend to cite more
influential and higher-quality research, which enhances the impact and
reputation of their scholarly contributions. These results underscore the
importance of considering citation practices when evaluating the scholarly
impact of authors and highlight the potential impact of journal quality on
citation patterns.

\section{Citation Network Clustering of Hyperprolific Authors}

A noteworthy observation was the number of authors with a high metric: the
top-ranked author has an h5-index of 87, 19 authors have an h5-index larger
than 60, 52 larger than 50, and 179 larger than 40. These achievements appear
to be even more difficult to explain than earlier observations of hyperprolific
authors. It was found in 2022 that works by authors with a high h5-index
(greater than 50) are more likely to be part of tightly-knit citation clusters
compared to other works with the same number of citations, which can be
explained by the fact that the clustering coefficients of a random sample of 50
works from authors with an h5-index larger than 50 appear to be significantly
different from a random sample of other works of the same size with the same
number of citations for each one (median 0.03). This highlights the
interconnected nature of highly cited authors' research within their respective
fields \cite{Spi23g}.

Similarly, we explored the phenomenal productivity and impact exhibited by the
authors at the top of the distribution (for the publications that were included
in the top 20\% of the journals by rank) for the adjusted h-indexes by
examining the clustering coefficient of the graph induced by incoming and
outgoing citations of distance 2 for a given work. This was done for all five
subject areas, and the results are shown in Tables \ref{tab:clustering_h5_top},
\ref{tab:clustering_jif_top}. In more detail, we used a random sample of 50
works from authors across all 5 subject areas, (meaning random 50 works per
subject) with an adjusted h-index larger than 30 and compared them to a random
sample of works that were publiced in the bottom 50\% of the journals by rank
for the adjusted h-indexes, with the same number of citations for each one. The
results are shown in Tables \ref{tab:clustering_h5_random},
\ref{tab:clustering_jif_random}.

\begin{table}[H]
    \centering
    \renewcommand{\arraystretch}{1.5}
    \begin{tabular}{|>{\centering\arraybackslash}m{4cm}|>{\centering\arraybackslash}m{3cm}|>{\centering\arraybackslash}m{3cm}|>{\centering\arraybackslash}m{3cm}|>{\centering\arraybackslash}m{3cm}|}
        \hline
        \textbf{Subject Area}         & \textbf{Median Clustering Coefficient (Top Authors)} & \textbf{Median Clustering Coefficient (Other Authors)} & \textbf{Mann-Whitney U} & \textbf{p-value} \\
        \hline
        Health Sciences               &                                                      &                                                        &                         & 0.0990           \\
        \hline
        Life Sciences                 &                                                      &                                                        &                         & 0.0188           \\
        \hline
        Multidisciplinary             &                                                      &                                                        &                         & 0.0050           \\
        \hline
        Physical Sciences             &                                                      &                                                        &                         & 0.0006           \\
        \hline
        Social Sciences \& Humanities &                                                      &                                                        &                         & 0.0007           \\
        \hline
    \end{tabular}
    \caption{Clustering Coefficients of Top Authors, publishing in the top 20\% of journals, and Authors publishing in the bottom 50\% of the journals by rank for Adjusted H-Index using H5}
    \label{tab:clustering_h5_top}
\end{table}

\begin{table}[H]
    \centering
    \renewcommand{\arraystretch}{1.5}
    \begin{tabular}{|>{\centering\arraybackslash}m{4cm}|>{\centering\arraybackslash}m{3cm}|>{\centering\arraybackslash}m{3cm}|>{\centering\arraybackslash}m{3cm}|>{\centering\arraybackslash}m{3cm}|}
        \hline
        \textbf{Subject Area}         & \textbf{Median Clustering Coefficient (Top Authors)} & \textbf{Median Clustering Coefficient (Other Authors)} & \textbf{Mann-Whitney U} & \textbf{p-value} \\
        \hline
        Health Sciences               &                                                      &                                                        &                         & 0.0990           \\
        \hline
        Life Sciences                 &                                                      &                                                        &                         & $<0.001$         \\
        \hline
        Multidisciplinary             &                                                      &                                                        &                         & 0.0050           \\
        \hline
        Physical Sciences             &                                                      &                                                        &                         & 0.0006           \\
        \hline
        Social Sciences \& Humanities &                                                      &                                                        &                         & 0.0007           \\
        \hline
    \end{tabular}
    \caption{Clustering Coefficients of Top Authors, publishing in the top 20\% of journals, and Authors publishing in the bottom 50\% of the journals by rank for Adjusted H-Index using JIF}
    \label{tab:clustering_jif_top}
\end{table}

After examining the clustering coefficients of the top authors and a random
sample of other authors, it can be deduced that the top authors have generally
a significantly higher clustering coefficient compared to other authors. This
suggests that the top authors are more likely to be part of tightly-knit
citation clusters, indicating a higher level of interconnectedness between
their works. This observation is consistent across all five subject areas, with
the differences being statistically significant in most cases.

In the \textbf{Health Sciences}, the median clustering coefficient for top
authors is 0.0427, compared to 0.0296 for other authors, indicating a higher
level of interconnectedness among the top authors, though the p-value of 0.0990
suggests this difference is not statistically significant. However, the trend
is clear and aligns with the pattern observed in other disciplines.

In the \textbf{Life Sciences}, the median clustering coefficient for top
authors is 0.0467, significantly higher than the 0.0286 observed for other
authors, with a p-value of 0.0188. This statistically significant difference
underscores the greater interconnectedness of top authors' works in this field.

The \textbf{Multidisciplinary} category shows a median clustering coefficient
of 0.0418 for top authors, compared to 0.0165 for other authors, with a highly
significant p-value of 0.0050. This substantial difference highlights the
tightly-knit nature of citation clusters among top authors in multidisciplinary
research.

In the \textbf{Physical Sciences}, the median clustering coefficient for top
authors is 0.0582, compared to 0.0306 for other authors, with a p-value of
0.0006, indicating a very significant difference. This suggests that top
authors in physical sciences are part of highly interconnected citation
networks.

In the \textbf{Social Sciences \& Humanities}, the median clustering
coefficient for top authors is 0.0567, while for other authors it is 0.0212,
with a p-value of 0.0007. This highly significant difference reinforces the
trend of top authors being part of tightly-knit citation clusters.
