\chapter{Results}
\label{ch:results}

\section{Comparing the H-index with the Adjusted H-indexes}

To evaluate the effectiveness of the adjusted h-indexes, we calculated the rank order correlation
between the h-index and the adjusted h-indexes from the publications that were included in the
top 20\% of the journals by rank. The Pearson, Spearman, and Kendall tau rank order correlations
were calculated for the h-index and the adjusted h-indexes. The results are shown in Table
\ref{tab:correlations}.





The clustering coefficients of a random sample of 50 works from authors with an
h5-index larger than 50 (median coefficient 0.05) appear to be significantly different from a random
sample of other works of the same size with the same number of citations for each one
(median 0.03).


Present your research results in terms of statistical findings,
empirical evaluation or other methods you have used in your research.
