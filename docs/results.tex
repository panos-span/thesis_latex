\chapter{Results}
\label{ch:results}

\section{Statistical Analysis of the Adjusted H-indexes}

To evaluate the relationship of the adjusted h-indexes with the traditional
h-index, we calculated the rank order correlation between the h-index and the
adjusted h-indexes from the publications that were included in the top 20\% of
the journals by rank. The Pearson, Spearman, and Kendall tau rank order
correlations were calculated for the h-index and the adjusted h-indexes using
the \emph{scipy.stats} package in Python using Python 3.11.9. The results are
shown in Table \ref{tab:correlations}.

\begin{table}[H]
    \centering
    \renewcommand{\arraystretch}{1.5}
    \begin{tabular}{|>{\centering\arraybackslash}m{5cm}|>{\centering\arraybackslash}m{3cm}|>{\centering\arraybackslash}m{3cm}|>{\centering\arraybackslash}m{3cm}|}
        \hline
        \textbf{Correlation}                                    & \textbf{Pearson} & \textbf{Spearman} & \textbf{Kendall tau} \\
        \hline
        \textbf{h-index vs. Adjusted h-index using H5}          & 0.686            & 0.693             & 0.637                \\
        \hline
        \textbf{h-index vs. Adjusted h-index using JIF}         & 0.667            & 0.678             & 0.616                \\
        \hline
        \textbf{h-index vs. Adjusted h-index using EigenFactor} & 0.680            & 0.688             & 0.629                \\
        \hline
    \end{tabular}
    \caption{Correlation between h-index and adjusted h-indexes}
    \label{tab:correlations}
\end{table}

From these results, we can see that the adjusted H-Indexes are significantly
correlated with the traditional H-Index, all having a $p-value$ of $0$, with
correlation coefficients in the range of 0.6 to 0.7. This indicates a strong
positive relationship between the traditional H-Index and the adjusted
H-Indexes, suggesting that researchers who rank highly according to the
traditional H-Index also tend to rank highly according to the adjusted
H-Indexes. More specifically, the Pearson correlation coefficients indicate a
strong linear relationship, while the Spearman correlation coefficients suggest
a strong monotonic relationship. Lastly, the Kendall tau correlation indicate a
strong ordinal relationship between the traditional H-Index and the adjusted
H-Indexes.

Compared to other H-Index variants, as reviewed in the literature, the
correlation is not perfect. This imperfection highlights the distinctiveness of
the adjusted H-Indexes. While the strong correlation signifies that the
adjusted H-Indexes retain the fundamental characteristics of the traditional
H-Index, the differences suggest that the adjustments made for journal quality
and subject specificity introduce meaningful variations. These variations
likely reflect the impact of publishing in higher-quality journals and the
influence of subject-specific citation practices, which are not captured by the
traditional H-Index.

To further investigate the relationship between the traditional H-Index and the
adjusted H-Indexes, we calculated the number of authors with adjusted H-indexes
and H-index above certain thresholds, along with their maximum values.
The results are shown in Table \ref{tab:thresholds}, where we can see that the
adjusted H-Indexes reduce the number of authors with high H-indexes compared to
the traditional H-Index, leading to disinflation of the H-index values and
potentially to a more accurate representation of the authors' scholarly impact.

\begin{table}[H]
    \centering
    \renewcommand{\arraystretch}{1.5}
    \begin{tabular}{|>{\centering\arraybackslash}m{2.5cm}|>{\centering\arraybackslash}m{2.5cm}|>{\centering\arraybackslash}m{2.5cm}|>{\centering\arraybackslash}m{2.5cm}|>{\centering\arraybackslash}m{2.5cm}|}
        \hline
        \textbf{Threshold} & \textbf{Adjusted H-Index using H5} & \textbf{Adjusted H-Index using JIF} & \textbf{Adjusted H-Index using EigenFactor} & \textbf{Traditional H-Index} \\
        \hline
        $>40$              & 179                                & 179                                 & 179                                         & 213                          \\
        \hline
        $>50$              & 52                                 & 52                                  & 52                                          & 62                           \\
        \hline
        $>60$              & 19                                 & 19                                  & 19                                          & 20                           \\
        \hline
        \emph{Max}         & 87                                 & 87                                  & 87                                          & 98                           \\
        \hline
    \end{tabular}
    \caption{Number of authors with adjusted h-indexes and traditional h-index above certain thresholds}
    \label{tab:thresholds}
\end{table}

\section{Citation Practices of Authors}

In the process of analyzing citation practices, it is essential to compare
different groups of authors to understand their impact and engagement within
the academic community. This analysis focuses on two distinct groups: top
authors publishing in lower-tier journals and randomly selected authors
publishing in top-tier journals. By examining the h-index of the journals these
authors cite, we aim to uncover citation practises and provide insights into
the quality and influence of their scholarly work.

Specifically, we selected the top 50 authors per subject publishing in the
bottom 50\% of journals by rank for the H5 and JIF journal rankings, where as
for the EigenFactor journal ranking we selected the same authors
publishing in the bottom 20\% of journals, and matched them with random
authors publishing in the top 20\% of journals, with matching h-indexes, where
the H5 calculations are stored in the tables \emph{orcid\_h5\_bottom} and
\emph{orcid\_h5\_filtered} accordingly. We then ensured one to one matching
based on the row number and calculated the average h-index of the cited
journals for each author in both groups.

Being done with the above extraction we proceeded to obtain the Mann-Whitney U
statistic and p-value for the comparison of the average h-index of the cited
journals between the two groups per subject area. The results are shown in
tables \ref{tab:citation_practices_jif}, \ref{tab:citation_practices_h5} and
\ref{tab:citation_practices_eigenfactor}.

Important to keep in mind is that the strictness of the journal filetring based
on the ranking method (H5, JIF, EF) varies, with the authors h-index
calculated based on the bottom 50\% of journals by rank for the H5 and JIF
journal rankings being of similar distribution (with 10 and 17 maximum h-index
values respectively). However, the authors h-index calculated based on the bottom
20\% of journals by rank for the EigenFactor journal ranking has a more differentiable
distribution (with a maximum h-index value of 26), thus seemingly being less
strict in the journal filtering process compared to the H5 and JIF journal
rankings. This difference is particularly evident when comparing the number of
authors with $h5 >= 10$, which is 3502 for the EigenFactor journal ranking, 7 for
the JIF journal ranking, and 3 for the H5 journal ranking, proving even further
the notable differences in the distribution of the journal rankings.

% TODO: Explain why this can also be seen be the number of authors with h5 >=10
%% EF: 3502 , JIF: 7 , H5: 3


\begin{table}[H]
    \centering
    \renewcommand{\arraystretch}{1.5}
    \begin{tabular}{|>{\centering\arraybackslash}m{3.8cm}|>{\centering\arraybackslash}m{2.5cm}|>{\centering\arraybackslash}m{2.5cm}|>{\centering\arraybackslash}m{2.5cm}|>{\centering\arraybackslash}m{2.5cm}|}
        \hline
        \textbf{Subject Area}         & \textbf{Average for Top Authors} & \textbf{Average for Random Authors} & \textbf{Mann-Whitney U} & \textbf{P-Value} \\
        \hline
        Health Sciences               & 81.913                           & 88.812                              & 993                     & 0.077            \\
        \hline
        Life Sciences                 & 52.945                           & 85.091                              & 598                     & $<0.001$         \\
        \hline
        Multidisciplinary             & 30.064                           & 90.346                              & 191                     & $<0.001$         \\
        \hline
        Physical Sciences             & 44.642                           & 90.983                              & 154                     & $<0.001$         \\
        \hline
        Social Sciences \& Humanities & 26.703                           & 64.529                              & 294                     & $<0.001$         \\
        \hline
    \end{tabular}
    \caption{Citation Practices of Top Authors from Lower-Tier Journals and Random Authors from Top-Tier Journals by Subject Area, with ranking of the journals based on JIF}
    \label{tab:citation_practices_jif}
\end{table}

\begin{table}[H]
    \centering
    \renewcommand{\arraystretch}{1.5}
    \begin{tabular}{|>{\centering\arraybackslash}m{3.8cm}|>{\centering\arraybackslash}m{2.5cm}|>{\centering\arraybackslash}m{2.5cm}|>{\centering\arraybackslash}m{2.5cm}|>{\centering\arraybackslash}m{2.5cm}|}
        \hline
        \textbf{Subject Area}         & \textbf{Average for Top Authors} & \textbf{Average for Random Authors} & \textbf{Mann-Whitney U} & \textbf{P-Value} \\
        \hline
        Health Sciences               & 62.629                           & 87.214                              & 615                     & $<0.001$         \\
        \hline
        Life Sciences                 & 49.607                           & 81.686                              & 371                     & $<0.001$         \\
        \hline
        Multidisciplinary             & 41.781                           & 88.819                              & 224                     & $<0.001$         \\
        \hline
        Physical Sciences             & 42.835                           & 83.658                              & 204                     & $<0.001$         \\
        \hline
        Social Sciences \& Humanities & 32.427                           & 77.161                              & 353                     & $<0.001$         \\
        \hline
    \end{tabular}
    \caption{Citation Practices of Top Authors from Lower-Tier Journals and Random Authors from Top-Tier Journals by Subject Area, with ranking of the journals based on H5}
    \label{tab:citation_practices_h5}
\end{table}

\begin{table}[H]
    \centering
    \renewcommand{\arraystretch}{1.5}
    \begin{tabular}{|>{\centering\arraybackslash}m{3.8cm}|>{\centering\arraybackslash}m{2.5cm}|>{\centering\arraybackslash}m{2.5cm}|>{\centering\arraybackslash}m{2.5cm}|>{\centering\arraybackslash}m{2.5cm}|}
        \hline
        \textbf{Subject Area}         & \textbf{Average for Top Authors} & \textbf{Average for Random Authors} & \textbf{Mann-Whitney U} & \textbf{P-Value} \\
        \hline
        Health Sciences               & 80.666                           & 71.348                              & 984                     & 0.077            \\
        \hline
        Life Sciences                 & 68.964                           & 70.534                              & 918                     & 0.258            \\
        \hline
        Multidisciplinary             & 57.497                           & 127.508                             & 191                     & $<0.001$         \\
        \hline
        Physical Sciences             & 73.885                           & 97.710                              & 387                     & $<0.001$         \\
        \hline
        Social Sciences \& Humanities & 43.236                           & 66.422                              & 318                     & $<0.001$         \\
        \hline
    \end{tabular}
    \caption{Citation Practices of Top Authors from Lower-Tier Journals and Random Authors from Top-Tier Journals by Subject Area, with ranking of the journals based on EigenFactor}
    \label{tab:citation_practices_eigenfactor}
\end{table}

The results of this analysis reveal significant differences in the citation
practices between the two groups of authors. The overall comparison shows that
the random authors, who publish in journals ranked in the top 20\%, have a
substantially higher average h-index of cited journals compared to the top
authors from the \emph{orcid\_h5\_bottom} dataset. Specifically, the overall bottom
average h-index median is 47.91, 47.50 and 69.02 (by H5 and JIF journal ranking
accordingly), while the overall random average h-index median is 82.9, 84.85
and 76.4. The Mann-Whitney U test statistic is 8931.0, 11842.0 and 14999.5
with both an extremely low p-value of $<0.001$, indicating a highly
statistically significant difference between the two groups.

The Mann-Whitney U test is a non-parametric test used to determine whether
there is a significant difference between the distributions of two independent
samples \cite{mann1947test}. In this context, the U value quantifies the
difference between the two groups. A higher U value generally indicates a
greater difference between the groups. The extremely low p-value associated
with the U value suggests that the observed difference is not due to random
chance, and there is a statistically significant difference between the two
groups.

Examining the results by subject by JIF journal ranking, a similar trend is
observed, with the exception of the Health Sciences, where the difference is
only marginally significant ($p-value = 0.077 > 0.05$). In contrast, all 
other subject areas show highly significant
differences in the average h-index of cited journals between the two groups,
with p-values ranging from $7.08 \times 10^{-6}$ to $2.86 \times 10^{-13}$.
This suggests that top authors from lower-tier journals tend to cite journals
with lower h-indexes compared to random authors from top-tier journals,
indicating differences in the quality and impact of the publications they
reference.

Similarly, the results by subject by H5 journal ranking show a resembling pattern,
but this time all subject areas show highly significant differences in the
average h-index of cited journals between the two groups, with p-values ranging
from $1.22 \times 10^{-5}$ to $5.70 \times 10^{-13}$. The difference between
the results of the two groups can be explained by the differences in
distribution of the H5 and JIF journal rankings, which influences the journals
that are considered high-quality and influential in each group, along with the
respective h-index values of the authors publishing in high and low tier
journals.

Lastly, the results from the EigenFactor journal ranking reveal additional
insights. For Health Sciences, the difference in median h-index is not
statistically significant with a $p-value > 0.05$, as well as for Life
Sciences. However, for the rest of the subjects, the differences are highly
significant with p-values ranging from $4.76 \times 10^{-9}$ to $3.60 \times
    10^{-6}$, further highlighting the noticeable differences in citation
practices between top authors from lower-tier and top-tier journals.

These findings suggest that authors publishing in lower-tier journals have
different citation practices compared to those publishing in top-tier journals.
The lower average h-index of cited journals among top authors from lower-tier
journals reflects a preference for citing less influential or lower-quality
research, potentially affecting the overall impact and credibility of their
work. In contrast, random authors from top-tier journals tend to cite more
influential and higher-quality research, which enhances the impact and
reputation of their scholarly contributions. These results underscore the
importance of considering citation practices when evaluating the scholarly
impact of authors and highlight the potential impact of journal quality on
citation patterns.

Important to acknowledge is that, the observed differences in the stringency of the journal ranking methods
play a crucial role in our analysis. The EigenFactor ranking method appears to
be less strict compared to the H5 and JIF rankings. This means that the
EigenFactor allows a broader range of journals to be considered lower-tier,
which could explain why some subjects showed non-significant differences in
citation practices. The H5 ranking, being the strictest, ensures that only the
highest-quality journals are included in the top 20\%, as well as for the lower
quality journals in the bottom 50\% of journals, thus showing more pronounced
differences in citation practices. Understanding these nuances is essential for
accurately interpreting the results and their implications for bibliometric
evaluations.

\section{Citation Network Clustering of Hyperprolific Authors}

A noteworthy observation was the number of authors with a high metric for the
traditional h5-index: the top-ranked author has an h5-index of 98, 20 authors
have larger than 60, 62 larger than 50, and 213 larger than 40. These
achievements appear to be even more difficult to explain than earlier % TODO: CHECK!
observations of hyperprolific authors. It was found in 2022 that works by
authors with a high h5-index (greater than 50) are more likely to be part of
tightly-knit citation clusters compared to other works with the same number of
citations, which can be explained by the fact that the clustering coefficients
of a random sample of 50 works from authors with an h5-index larger than 50
appear to be significantly different from a random sample of other works of the
same size with the same number of citations for each one (median 0.03). This
highlights the interconnected nature of highly cited authors' research within
their respective fields \cite{Spi23g}.

Similarly, we explored the phenomenal productivity and impact exhibited by the
authors at the top of the distribution by examining the clustering coefficient
of the graph induced by incoming and outgoing citations of distance 2 for a
given work. However this time, it was done for all five subject areas, and the
results are shown in Tables \ref{tab:clustering_top_random}. In more detail, we
used a random sample of 50 works from the top 15 authors of each subject area,
meaning we got 50 works from each subject area, and compared them to a random
sample of other works of the same size with the same number of citations for
each one. % TODO: CHECK!!

\begin{table}[H]
    \centering
    \renewcommand{\arraystretch}{1.5}
    \begin{tabular}{|>{\centering\arraybackslash}m{3.8cm}|>{\centering\arraybackslash}m{2.5cm}|>{\centering\arraybackslash}m{2.5cm}|>{\centering\arraybackslash}m{2.5cm}|>{\centering\arraybackslash}m{2.5cm}|}
        \hline
        \textbf{Subject Area}         & \textbf{Median Clustering Coefficient (Top Authors)} & \textbf{Median Clustering Coefficient (Other Authors)} & \textbf{Mann-Whitney U} & \textbf{P-value} \\
        \hline
        Health Sciences               & 0.036                                                & 0.031                                                  & 1218.0                  & $<0.05$          \\
        \hline
        Life Sciences                 & 0.045                                                & 0.027                                                  & 1390.0                  & $<0.005$         \\
        \hline
        Multidisciplinary             & 0.062                                                & 0.031                                                  & 1393.0                  & $<0.001$         \\
        \hline
        Physical Sciences             & 0.044                                                & 0.026                                                  & 1646.5                  & $<0.001$         \\
        \hline
        Social Sciences \& Humanities & 0.056                                                & 0.036                                                  & 1533.0                  & 0.005            \\
        \hline
    \end{tabular}
    \caption{Comparison of Median Clustering Coefficient between Top Authors and Other Authors across Different Subject Areas}
    \label{tab:clustering_top_random}
\end{table}

After examining the clustering coefficients of the top authors publications and a random
sample of other authors publications, it can be deduced that the top authors have generally
a significantly higher clustering coefficient compared others. This
suggests that the top authors are more likely to be part of tightly-knit
citation clusters, indicating a higher level of interconnectedness between
their works. This observation is consistent across all five subject areas, with
the differences being statistically significant in all cases, however the
differences are more pronounced in some subject areas than others.

In the Health Sciences, the median clustering coefficient of top authors works
(0.036) compared to other authors works (0.031) with a p-value of 0.045, shows a
statistically significant but modest difference. This suggests that while there
is a higher degree of interconnectedness, it is not as pronounced as in other
fields. Life Sciences show a more pronounced difference, with top authors works
having a median clustering coefficient of 0.045 versus 0.027 for other authors works
and a highly significant p-value of 0.004, which indicates a strong
interconnectedness in this field, likely due to the collaborative nature of
life sciences research.

The Multidisciplinary field exhibits the highest difference with a median
clustering coefficient of 0.062 for top authors works compared to 0.031 for others works,
and a p-value of less than 0.001. This noticeable disparity underscores the
broad and influential impact of top authors in multidisciplinary research,
where diverse collaboration and citation networks are more prevalent.
Similarly, in Physical Sciences, the median clustering coefficient for top
authors (0.044) versus other authors (0.026) and a p-value of less than 0.001,
suggests that top authors in this field are significantly more interconnected,
reflecting the collaborative nature of physical sciences research.

Lastly, the Social Sciences and Humanities show a median clustering coefficient
of 0.056 for top authors works compared to 0.036 for other authors works, with a p-value of
0.005. This significant difference indicates that top authors in this field are
also part of tightly-knit citation clusters, reflecting the interconnected
nature of research within social sciences and humanities disciplines.

Overall, the above results underscore the importance of considering citation
network structures when evaluating the impact and influence of researchers. The
higher clustering coefficients for top authors suggest that their work not only
garners more citations but also forms the backbone of influential and
collaborative research networks within their fields.
